\chapter{Introduction}

Is Florida getting warmer? The answer must most likely be 'yes' in reality, based on common knowledge. How do we get the answer? We need to do a hypothesis testing. H1 (alternative hypothesis) is what we want to get the conclusion, H0 (the null hypothesis) is the opposite. Therefore, the H1 is 'Florida is getting warmer', and H0 is 'Florida is not getting warmer'. 


\cite{greenwade93}

\section{Objectives}


\section{Challenges}

\section{Contributions}



\chapter{Results}

\section{Correlation Coefficient  Formula}

    \begin{equation} 
        r = \frac{ \sum_{i=1}^{n}(x_i-\bar{x})(y_i-\bar{y}) }{%
        \sqrt{\sum_{i=1}^{n}(x_i-\bar{x})^2}\sqrt{\sum_{i=1}^{n}(y_i-\bar{y})^2}}
    \end{equation}

\section{Correlation Coefficient Result}

Correlation coefficient between years and temperature, is 0.5331784. 

\section{Histogram}
(Needs later when the plot can be seen in VSCode Ubuntu computer. )

% starting copied from https://github.com/joobaloo/CMEEcoursework/blob/master/week3/code/florida_writeup.tex
\begin{figure}[h!]
    \begin{center}
        \includegraphics[width=90mm]{../results/florida_histogram.png}
    \end{center}
    \caption{Observed Correlation Compared With Random Permutations.}
    \label{fig1}
\end{figure}
% ended copied from https://github.com/joobaloo/CMEEcoursework/blob/master/week3/code/florida_writeup.tex

\chapter{Interpretation}

\section{Correlation Coefficient}

Don't know yet







